\subsection{Design of a solar system}
We dispose of some classes both in C++ and JAVA that allow building applications working on
notions of planets, stars and solar systems.

		\subsubsection{Planet}

This class represents the characteristics of a planet.

		\begin{tableiii}{l|l|l}{.5}{Type}{Attribute name}{Description}
\lineiii{double}{diameter}{the average diameter of the planet  				}
\end{tableiii}


		\begin{itemize}
\item function double \textbf{getDistanceToSun}(int \textbf{iDay}, int \textbf{iMonth}, int \textbf{iYear})
				


This function returns the distance to the sun at a given trivial earthly date. This function
reclaims more attributes for the planet, but we'll see it later (I'm afraid not!).

			
\end{itemize}


		\subsubsection{Earth}

This class represents our planet, for instantiating our particular solar system for instance,
and working on geopolitical data perhaps!

		\begin{tableiii}{l|l|l}{.5}{Type}{Attribute name}{Description}
\lineiii{std::vector<std::string>}{countryNames}{the name of all countries are put into  				}
\end{tableiii}


		\subsubsection{SolarSystem}

This class represents the solar system, with its constituents, the sun excluded for the
moment.

		\begin{tableiii}{l|l|l}{.5}{Type}{Attribute name}{Description}
\lineiii{std::vector<Planet*>}{planets}{the planets that compose the solar system.  				}
\end{tableiii}


	